% ===========================================================================================
% This is an example LaTeX file that uses the custom "bunne" beamer theme
%
% @author pmrobbe
% @date 09/14/23

\documentclass[aspectratio=169, 12pt, t]{beamer}

% ===========================================================================================
% Title, date and authors

% A title that spans multiple lines
\title{An extremely long title that \\ doesn't fit on a single line}

% Use the institute command to add conference name and date, location, ...
\institute{Some awesome conference}

% Authors and affiliations in a tabular environment
\author{%
    \hspace{-5pt}%
    \begin{tabular}{ccc}
        Author One   & 
        Author Two   & 
        Author Three \\
        \includegraphics[height=1.2cm]{plt/AffiliationOne} &
        \includegraphics[height=1.2cm]{plt/AffiliationTwo} &
        \includegraphics[height=1.2cm]{plt/AffiliationThree} \\
    \end{tabular}
}

% ===========================================================================================
% Select beamer style
\usepackage{theme/beamerthemebunne}

% ===========================================================================================
% Preamble

% With this option, only equations that are references in the text get a label
\usepackage{mathtools}
    \mathtoolsset{showonlyrefs}

% We need this for a good looking arrow head
\usetikzlibrary{arrows.meta}

% By default all math in TikZ nodes are set in inline mode
% This line avoids that we have to pput \displaystyle before every equation
\everymath{\displaystyle}

% ===========================================================================================
% Contents begins here
\begin{document}

% ===========================================================================================
% Title slide
\titleslide{%
}

% ===========================================================================================
% Funding slide
% \fundingslide{%
% }

% ===========================================================================================
% Section: Introduction
\section{Introduction}

\outlineslide{currentsection, hidesubsections}

\begin{frame}
    \frametitle{Credit}
    This is a \texttt{beamer} theme inspired by the slides for the ICML 2023 tutorial on \emph{Optimal Transport in Learning, Control, and Dynamical Systems} by C. Bunne and M. Cuturi, available at
    \begin{center}
        \url{https://icml.cc/virtual/2023/tutorial/21559}
    \end{center}
\end{frame}

\begin{frame}
    \frametitle{Key features}
    \begin{coloreditemize}
        \item A colored label on the top right indicating the section number, cycling automatically between \tikz{\node[fill=color1!50] {};}, \tikz{\node[fill=color2!50] {};} and \tikz{\node[fill=color3!50] {};}
        \item Corresponding coloring of \texttt{item}'s in \texttt{itemize} and \texttt{enumerate} environments
        \item Outline slides with custom highlighting
        \item Special math highlighting, e.g.,
        \begin{equation}
                \hlmath{color3!30}{\int_a^bf(x)dx = \lim_{n\rightarrow\infty} \sum_{j=1}^{n} f(x_j) \hlmath{color1!50}{\Delta x_j}}
        \end{equation}
        \item A custom citation style~\cite{verdu2017individual}
    \end{coloreditemize}
\end{frame}

% ===========================================================================================
% Section: Main usage
\section{Main usage}

\outlineslide{currentsection, hideothersubsections}

\subsection{Colored itemizations and enumarations}

\begin{frame}
    \frametitle{Unnumbered lists}
    Here's an example of a bullet list with \texttt{itemize}
    \begin{itemize}
        \item This is the first bullet
        \item This is the second bullet
        \begin{itemize}
            \item This is a subbullet
            \begin{itemize}
                \item This is a subsubbullet
            \end{itemize}
        \end{itemize}
        \item This is the third bullet
    \end{itemize}
    \hl{color3!30}{%
        Use \texttt{coloreditemize} to create colored bullet lists
        \begin{coloreditemize}
            \item This is the first bullet
            \item This is the second bullet
        \end{coloreditemize}
    }
\end{frame}

\begin{frame}[t]
    \frametitle{Numbered lists}
    Use \texttt{enumerate} for numbered lists
    \begin{enumerate}
        \item This is the first item
        \item This is the second item
        \item This is the third item
    \end{enumerate}
    The background color of the \texttt{item} changes depending on the section number\\[\baselineskip]
    A nested numbered and unnumbered list
    \begin{enumerate}
        \item This is the first item
        \begin{coloreditemize}
            \item This is a first subbullet
            \item This is a second subbullet
        \end{coloreditemize}
        \item This is the second item
    \end{enumerate}
\end{frame}

\subsection{Outline slides}

\begin{frame}[fragile]
    \frametitle{Outline slides}
    To add an outline slide, use \verb!\outlineslide{<options>}! \\[\baselineskip]
    \textbf{Note}: \verb!<options>! will be passed on to \verb!\tableofcontents[<options>]{}!, so to add an outline slide which highlights the current section, you can use \verb!\outlineslide{currentsection, hideothersubsections}!
\end{frame}

\subsection{Highlighting math}

\begin{frame}[fragile]
    \frametitle{Highlighting math}
    To highlight math, use the \verb!\hlmath{<color>}{<math>}! command
    \vskip10ex%
    \begin{flushright}
        \begin{minipage}{0.5\textwidth}
            \begin{equation}
                \displaystyle\hlmath{color3!20}{T^\star = \underset{T:T_\#\mu\approx\nu}{\mathrm{argmin}} \int_{\mathbb{R}^d}\|x - \tikz[remember picture]{\coordinate (T);}\hlmath{color1!50}{T}(x)\|_2^2d\mu(x)\tikz[remember picture]{\coordinate (n);}}
            \end{equation}
            \begin{tikzpicture}[remember picture, overlay]
                \node[anchor=east] (euclid) at ([yshift=7ex]n) {Euclidean distance as transport cost};
                \coordinate (begin) at (euclid.351);
                \coordinate (end) at ([yshift=-3ex]begin);
                \draw[very thick, ->, >={Stealth[length=2mm, width=1.75mm]}] (begin) -- (end);
            \end{tikzpicture}
        \begin{tikzpicture}[remember picture, overlay]
            \draw[draw=color1, very thick] ([yshift=-.5ex]T) -- ++(-0.75, -0.75) -- node[pos=1, anchor=south west] {\textcolor{color1}{\textbf{Approach:}} learn $T$} ++(-10.25, 0);
        \end{tikzpicture}
        \end{minipage}%
    \end{flushright}
\end{frame}

% ===========================================================================================
% Section: Conclusion
\section{Conclusion}

\outlineslide{currentsection, hideothersubsections}

\begin{frame}
    \frametitle{Adding citations}
    \begin{enumerate}
        \item At the end of nineteenth century, approximately thirty nearly complete skeletons of the ornithopod dinosaur Iguanodon bernissartensis were discovered in Lower Cretaceous deposits in Bernissart (Belgium)~\cite{verdu2017individual}
        \item People simply do not know everything about everything. There are holes in their knowledge, gaps in their expertise \cite{dunning2011dunning}
        \item The lithologies of rocks along Pfeiffer Beach are typically Franciscan, consisting of sandstone, siltstone, mudstone, shale, conglomerate, greenstone, and chert \cite{underwood1977pfeiffer}
    \end{enumerate}
\end{frame}

% ===========================================================================================
% References
\bibliographyslide{}

\end{document}